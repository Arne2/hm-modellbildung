\documentclass[]{scrartcl}
\usepackage{lmodern}
\usepackage{amssymb,amsmath}
\usepackage{ifxetex,ifluatex}
\usepackage{fixltx2e} % provides \textsubscript
\ifnum 0\ifxetex 1\fi\ifluatex 1\fi=0 % if pdftex
  \usepackage[T1]{fontenc}
  \usepackage[utf8]{inputenc}
\else % if luatex or xelatex
  \ifxetex
    \usepackage{mathspec}
  \else
    \usepackage{fontspec}
  \fi
  \defaultfontfeatures{Ligatures=TeX,Scale=MatchLowercase}
\fi
% use upquote if available, for straight quotes in verbatim environments
\IfFileExists{upquote.sty}{\usepackage{upquote}}{}
% use microtype if available
\IfFileExists{microtype.sty}{%
\usepackage{microtype}
\UseMicrotypeSet[protrusion]{basicmath} % disable protrusion for tt fonts
}{}
\usepackage{hyperref}
\hypersetup{unicode=true,
            pdftitle={Angabe},
            pdfauthor={Team\ldots{}},
            pdfborder={0 0 0},
            breaklinks=true}
\urlstyle{same}  % don't use monospace font for urls
\IfFileExists{parskip.sty}{%
\usepackage{parskip}
}{% else
\setlength{\parindent}{0pt}
\setlength{\parskip}{6pt plus 2pt minus 1pt}
}
\setlength{\emergencystretch}{3em}  % prevent overfull lines
\providecommand{\tightlist}{%
  \setlength{\itemsep}{0pt}\setlength{\parskip}{0pt}}
\setcounter{secnumdepth}{5}
% Redefines (sub)paragraphs to behave more like sections
\ifx\paragraph\undefined\else
\let\oldparagraph\paragraph
\renewcommand{\paragraph}[1]{\oldparagraph{#1}\mbox{}}
\fi
\ifx\subparagraph\undefined\else
\let\oldsubparagraph\subparagraph
\renewcommand{\subparagraph}[1]{\oldsubparagraph{#1}\mbox{}}
\fi

\usepackage{graphicx}
\usepackage{array}
\usepackage{ragged2e}
\usepackage[section]{placeins}
\makeatletter
\AtBeginDocument{%
  \expandafter\renewcommand\expandafter\subsection\expandafter{%
    \expandafter\@fb@secFB\subsection
  }%
}
\makeatother
 
\title{Angabe 1}
\providecommand{\subtitle}[1]{}
\subtitle{Untertitel}
\author{Daniel Graf, Dimitrie Diez, Arne Schöntag, Peter Müller}
\date{}

\begin{document}
\maketitle


\tableofcontents

\section{Einführung}

% Problem -> Motivation

\section{Messexperiment}
\section{Überprüfung auf Normalverteilung}
\section{Modell}
\section{Lineare Regression}
\subsection{Prüfung auf eine Abhängigkeit}
\subsection{Mehrere Abhängigkeiten}
\subsection{Konditionierung}
Die Konditionierung des Problems wurde anhand der Desingmatrix des linearen Modells untersucht. Je näher ein ermittelter Wert an 1 liegt, desto besser ist das Problem konditioniert.

Bei der Untersuchung der Treppengeschwindigkeit aufwärts wurde ein Konditionierungswert von $744.683$ ermittelt. Auch bei der Betrachtung der Treppengeschwindigkeit abwärts wurde ähnliches Ergebnis von $654.274$ berechnet. Die Betrachtung zweier Parameter (Ebenengeschwindigkeit und Körpergröße) ergab sogar den Wert $3.27621*10^7$.
Diese schlechte Konditionierung erfolgt durch Rechenfehlern, die bei numerischen Berechnungen erfolgen. In diesen Berechnungen wurden 6 bis 7 ($6.1296$, $6.48353$) bzw. $17.3048$ Dezimalstellen verloren.

Wenn man die Konditionierung mit der QR-Zerlegung betrachtet, welche auch in Wirklichkeit von Mathematica verwendet wird, wurde eine wesentlich bessere Konditionierung ermittelt.

Die Betrachtung der Konditionierung der oberen Dreiecksmatrix R ergab bei der Treppengeschwindigkeit aufwärts nun einen Wert von $29.6338$. Hierbei wurden auch weniger verlorene Dezimalstellen festgestellt. Diese betrugen nur noch $3.38891$. Bei der Treppengeschwindigkeit abwärts wurde eine Konditionierung von $27.9171$ und $7164.48$ bei zwei Parametern (Verlorene Dezimalstellen bei $3.32924$ bzw. $8.87689$). Bei einem Parameter - der Ebenengeschwindigkeit - ist die Konditionierung nun deutlich besser. Für zwei Parameter (Körpergröße und Ebenengeschwindigkeit) lässt sich schließen, dass für diese Auswertung das Problem tatsächlich schlecht Konditioniert ist und die Aussagekraft der Ergebnisse bei dieser Untersuchung in Frage gestellt werden kann.

Dadurch, dass Mathematica die QR-Zerlegung benutzt sind diese Ergebnisse auch wesentlich aussagekräftiger als die erste Untersuchung der Konditionierung.



\section{Ergebnisse}
\section{Ermitteltes Modell}

\section{Vergleich mit Daten aus 2012}
\subsection{Überprüfung auf Normalverteilung}
\subsection{Lineare Regression}
\subsection{Vergleich}

\section{Verbund von alten und neuen Daten}

\section{Fazit}

\end{document}