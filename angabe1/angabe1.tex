\include{header}
 
\title{Angabe 1}
\providecommand{\subtitle}[1]{}
\subtitle{Untertitel}
\author{Daniel Graf, Dimitrie Diez, Arne Schöntag, Peter Müller}
\date{}

\begin{document}
\maketitle


\tableofcontents

\section{Einführung}
Da heutzutage das Laufverhalten von Menschen noch nicht vollständig erforscht ist, muss dieses durch Personenstromexperimente weiter untersucht werden. Vor allem sind das Laufverhalten und die Geschwindigkeit auf Treppen noch größtenteils unbekannt und werfen viele Fragen auf. Mit den gewonnenen Erkenntnissen dieser Untersuchungen ist es beispielsweise möglich, bei der Gebäudeplanung die Fluchtwege geeignet zu setzten. Eine schlechte Planung kann daher im Ernstfall schwere Folgen für die Insassen einer Einrichtung haben.

Diese Arbeit untersucht, ob ein möglicher Zusammenhang zwischen der Wunschgeschwindigkeit einer Person auf einer freien Fläche und der Wunschgeschwindigkeit auf einer Treppe. Dabei werden die Faktoren Körpergröße, Alter und Geschlecht betrachtet.
Dazu werden zwei Hypothesen diskutiert:
\begin{list}{-}{}
	\item Die Geschwindigkeit auf der Treppe hängt linear von der Wunschgeschwindigkeit ab.
	\item Es gibt keinen Zusammenhang der Geschwindigkeit auf der Treppe mit der auf der Ebene durch die Taktung durch die Stufen.
\end{list}
 
Zur Behandlung dieser Problematik wird ein Messexperiment mit Probanden durchgeführt. Auf diesem Experiment basieren die darauffolgenden Untersuchungen und Beobachtungen. Die Daten des Messexperiments werden zunächst auf Normalverteilung überprüft. Anschließend wird über lineare Regression überprüft, ob ein Zusammenhang zwischen Treppengeschwindigkeit und den Größen Wunschgeschwindigkeit, Körpergröße und Rundennummer besteht. Die daraus entstandenen Regressionsmodelle werden einem t-Test unterzogen, um zu überprüfen, ob tatsächlich ein linearer Zusammenhang besteht. Die gewonnenen Erkenntnisse dieses Experiments werden im Zusammenhang mit bereits gemessenen Daten aus 2012 verglichen und ausgewertet.

\section{Messexperiment}
\section{Überprüfung auf Normalverteilung}
\section{Modell}
\section{Lineare Regression}
\subsection{Prüfung auf eine Abhängigkeit}
\subsection{Mehrere Abhängigkeiten}
\subsection{Konditionierung}

\section{Ergebnisse}
\section{Ermitteltes Modell}

\section{Vergleich mit Daten aus 2012}
\subsection{Überprüfung auf Normalverteilung}
\subsection{Lineare Regression}
\subsection{Vergleich}

\section{Verbund von alten und neuen Daten}

\section{Fazit}

\end{document}