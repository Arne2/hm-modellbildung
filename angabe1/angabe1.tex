\include{header}
 
\title{Angabe 1}
\providecommand{\subtitle}[1]{}
\subtitle{Untertitel}
\author{Daniel Graf, Dimitrie Diez, Arne Schöntag, Peter Müller}
\date{}

\begin{document}
\maketitle


\tableofcontents

\section{Einführung}

% Problem -> Motivation

\section{Messexperiment}

Das Messexperiment wurde am $05.04.2017$ im Lichthof der Hochschule München (Lothstraße 64) durchgeführt. Es nahmen $22$ Probanden im Alter von $20-29$ Jahren teil. Das Experiment bestand aus drei Teilen. 

Zunächst wurde die Wunschgeschwindigkeit in der Ebene gemessen. Hierfür ging jeder Proband eine markierte Strecke von $27,3m$ ab und stoppte die hierfür benötigte Zeit. Anschließend wurde dieser Vorgang zweimal wiederholt und die entsprechende Rundennummer vermerkt. Im zweiten Teil erfolgte die Messung der benötigten Zeit für einen Treppenaufstieg. Die Treppenlänge betrug $9m$. Jeder Proband führte den Vorgang dreimal durch und vermerkte die benötigte Zeit und die entsprechende Rundennummer. Analog hierzu wurde im dritten Teil des Experiments die Zeit beim Treppenabstieg gemessen. 

Neben den gemessenen Zeiten in jeder Runde, dem Alter und der Körpergröße ist auch das Geschlecht jedes Probanden bekannt. Weitere Informationen sind in der beiliegenden Versuchsbeschreibung "Choreographie\_Treppengeschwindigkeit\_2017" aufgeführt. In den folgenden Kapiteln erfolgt die Auswertung der ermittelten Messwerte.

\section{Überprüfung auf Normalverteilung}
\section{Modell}
\section{Lineare Regression}
\subsection{Prüfung auf eine Abhängigkeit}
\subsection{Mehrere Abhängigkeiten}
\subsection{Konditionierung}

\section{Ergebnisse}
\section{Ermitteltes Modell}

\section{Vergleich mit Daten aus 2012}
\subsection{Überprüfung auf Normalverteilung}
\subsection{Lineare Regression}
\subsection{Vergleich}

\section{Verbund von alten und neuen Daten}

\section{Fazit}

\end{document}