\include{header}
 
\title{Angabe 1}
\providecommand{\subtitle}[1]{}
\subtitle{Untertitel}
\author{Daniel Graf, Dimitrie Diez, Arne Schöntag, Peter Müller}
\date{}

\begin{document}
\maketitle


\tableofcontents

\section{Einführung}
Da heutzutage das Laufverhalten von Menschen noch nicht 100\%tig vorhersehbar ist, muss dieses durch Personenstromexperimente weiter untersucht werden. Dabei ist das Laufverhalten auf Treppen noch größtenteils unbekannt und wirft viele Fragen auf. Mit den gewonnenen Erkenntnissen dieser Untersuchungen ist es beispielsweise möglich bei der Gebäudeplanung die Fluchtwege geeignet zu setzten.

Zu untersuchen ist ein möglicher Zusammenhang zwischen der Wunschgeschwindigkeit einer Person auf einer freien Fläche und der Wunschgeschwindigkeit auf einer Treppe. Es werden die Faktoren Körpergröße, Alter und Geschlecht betrachtet werden.
Dazu werden zwei Hypothesen diskutiert:
\begin{list}{-}{}
	\item Die Geschwindigkeit auf der Treppe hängt linear von der Wunschgeschwindigkeit ab.
	\item Es gibt keinen Zusammenhang der Geschwindigkeit auf der Treppe mit der auf der Ebene durch die Taktung durch die Stufen.
\end{list}
 
Zur Überprüfung der Hypothesen wird ein Experiment durchgeführt.
% Problem -> Motivation

\section{Messexperiment}
\section{Überprüfung auf Normalverteilung}
\section{Modell}
\section{Lineare Regression}
\subsection{Prüfung auf eine Abhängigkeit}
\subsection{Mehrere Abhängigkeiten}
\subsection{Konditionierung}

\section{Ergebnisse}
\section{Ermitteltes Modell}

\section{Vergleich mit Daten aus 2012}
\subsection{Überprüfung auf Normalverteilung}
\subsection{Lineare Regression}
\subsection{Vergleich}

\section{Verbund von alten und neuen Daten}

\section{Fazit}

\end{document}