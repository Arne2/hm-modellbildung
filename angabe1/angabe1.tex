\documentclass[]{scrartcl}
\usepackage{lmodern}
\usepackage{amssymb,amsmath}
\usepackage{ifxetex,ifluatex}
\usepackage{fixltx2e} % provides \textsubscript
\ifnum 0\ifxetex 1\fi\ifluatex 1\fi=0 % if pdftex
  \usepackage[T1]{fontenc}
  \usepackage[utf8]{inputenc}
\else % if luatex or xelatex
  \ifxetex
    \usepackage{mathspec}
  \else
    \usepackage{fontspec}
  \fi
  \defaultfontfeatures{Ligatures=TeX,Scale=MatchLowercase}
\fi
% use upquote if available, for straight quotes in verbatim environments
\IfFileExists{upquote.sty}{\usepackage{upquote}}{}
% use microtype if available
\IfFileExists{microtype.sty}{%
\usepackage{microtype}
\UseMicrotypeSet[protrusion]{basicmath} % disable protrusion for tt fonts
}{}
\usepackage{hyperref}
\hypersetup{unicode=true,
            pdftitle={Angabe},
            pdfauthor={Team\ldots{}},
            pdfborder={0 0 0},
            breaklinks=true}
\urlstyle{same}  % don't use monospace font for urls
\IfFileExists{parskip.sty}{%
\usepackage{parskip}
}{% else
\setlength{\parindent}{0pt}
\setlength{\parskip}{6pt plus 2pt minus 1pt}
}
\setlength{\emergencystretch}{3em}  % prevent overfull lines
\providecommand{\tightlist}{%
  \setlength{\itemsep}{0pt}\setlength{\parskip}{0pt}}
\setcounter{secnumdepth}{5}
% Redefines (sub)paragraphs to behave more like sections
\ifx\paragraph\undefined\else
\let\oldparagraph\paragraph
\renewcommand{\paragraph}[1]{\oldparagraph{#1}\mbox{}}
\fi
\ifx\subparagraph\undefined\else
\let\oldsubparagraph\subparagraph
\renewcommand{\subparagraph}[1]{\oldsubparagraph{#1}\mbox{}}
\fi

\usepackage{graphicx}
\usepackage{array}
\usepackage{ragged2e}
\usepackage[section]{placeins}
\makeatletter
\AtBeginDocument{%
  \expandafter\renewcommand\expandafter\subsection\expandafter{%
    \expandafter\@fb@secFB\subsection
  }%
}
\makeatother
 
\title{Angabe 1}
\providecommand{\subtitle}[1]{}
\subtitle{Untertitel}
\author{Daniel Graf, Dimitrie Diez, Arne Schöntag, Peter Müller}
\date{}

\begin{document}
\maketitle


\tableofcontents

\section{Einführung}
Da heutzutage das Laufverhalten von Menschen noch nicht 100\%tig vorhersehbar ist, muss dieses durch Personenstromexperimente weiter untersucht werden. Dabei ist das Laufverhalten und die Geschwindigkeit auf Treppen noch größtenteils unbekannt und wirft viele Fragen auf. Mit den gewonnenen Erkenntnissen dieser Untersuchungen ist es beispielsweise möglich bei der Gebäudeplanung die Fluchtwege geeignet zu setzten. Eine schlechte Planung kann daher im Ernstfall schwere Folgen für die Insassen einer Einrichtung haben.

Zu untersuchen ist ein möglicher Zusammenhang zwischen der Wunschgeschwindigkeit einer Person auf einer freien Fläche und der Wunschgeschwindigkeit auf einer Treppe. Es werden die Faktoren Körpergröße, Alter und Geschlecht betrachtet werden.
Dazu werden zwei Hypothesen diskutiert:
\begin{list}{-}{}
	\item Die Geschwindigkeit auf der Treppe hängt linear von der Wunschgeschwindigkeit ab.
	\item Es gibt keinen Zusammenhang der Geschwindigkeit auf der Treppe mit der auf der Ebene durch die Taktung durch die Stufen.
\end{list}
 
Zu Beginn wird das durchgeführte Messexperiment erläutert und die Resultate beschrieben. Auf diesem Experiment basieren die darauffolgenden Untersuchungen und Beobachtungen. Anschließend werden die Ergebnisse des Experiments auf Normalverteilung überprüft. Dabei werden alle durchgeführten Messungen untersucht. Im nächsten Abschnitt wird das Modell beschrieben. Sobald dieses Thema geklärt ist, können die Untersuchungen auf Lineare Regression begonnen werden. Dieser Teil beinhaltet die Überprüfungen, ob bzw. welche der Hypothesen stimmen oder nicht. Anschließen können die Ergebnisse dieser Untersuchungen betrachtet und erörtert werden.
Danach kann das daraus ermittelte Modell beschrieben werden.
Die gewonnenen Erkenntnisse dieser Messreihe sollen im Zusammenhang mit bereits gemessenen Daten aus 2012 verglichen und ausgewertet werden.
Wenn die alten und neuen Daten und Resultate verglichen wurden, können die Ergebnisse in einem Fazit zusammengefasst werden.

\section{Messexperiment}
\section{Überprüfung auf Normalverteilung}
\section{Modell}
\section{Lineare Regression}
\subsection{Prüfung auf eine Abhängigkeit}
\subsection{Mehrere Abhängigkeiten}
\subsection{Konditionierung}

\section{Ergebnisse}
\section{Ermitteltes Modell}

\section{Vergleich mit Daten aus 2012}
\subsection{Überprüfung auf Normalverteilung}
\subsection{Lineare Regression}
\subsection{Vergleich}

\section{Verbund von alten und neuen Daten}

\section{Fazit}

\end{document}