\include{header}
 
\title{Angabe 1}
\providecommand{\subtitle}[1]{}
\subtitle{Untertitel}
\author{Daniel Graf, Dimitrie Diez, Arne Schöntag, Peter Müller}
\date{}

\begin{document}
\maketitle


\tableofcontents

\section{Einführung}

% Problem -> Motivation

\section{Messexperiment}

Das Messexperiment wurde am $05.04.2017$ im Lichthof der Hochschule München (Lothstraße 64) durchgeführt. Es nahmen $22$ Probanden im Alter von $20-29$ Jahren teil. Das Experiment bestand aus drei Teilen. 

Zunächst wurde die Wunschgeschwindigkeit in der Ebene gemessen. Hierfür ging jeder Proband eine markierte Strecke von $27,3m$ ab und stoppte die hierfür benötigte Zeit. Anschließend wurde dieser Vorgang zweimal wiederholt und die entsprechende Rundennummer vermerkt. Im zweiten Teil erfolgte die Messung der benötigten Zeit für einen Treppenaufstieg. Die Treppenlänge betrug $9m$. Jeder Proband führte den Vorgang dreimal durch und vermerkte die benötigte Zeit und die entsprechende Rundennummer. Analog hierzu wurde im dritten Teil des Experiments die Zeit beim Treppenabstieg gemessen. 

Neben den gemessenen Zeiten in jeder Runde, dem Alter und der Körpergröße ist auch das Geschlecht jedes Probanden bekannt. Weitere Informationen sind in der beiliegenden Versuchsbeschreibung "Choreographie\_Treppengeschwindigkeit\_2017" aufgeführt. In den folgenden Kapiteln erfolgt die Auswertung der ermittelten Messwerte.

\section{Überprüfung auf Normalverteilung}
\section{Modell}
\section{Lineare Regression}
\subsection{Prüfung auf eine Abhängigkeit}
\subsection{Mehrere Abhängigkeiten}
\subsection{Konditionierung}
Die Konditionierung des Problems wurde anhand der Desingmatrix des linearen Modells untersucht. Je näher ein ermittelter Wert an 1 liegt, desto besser ist das Problem konditioniert.

Bei der Untersuchung der Treppengeschwindigkeit aufwärts wurde ein Konditionierungswert von $744.683$ ermittelt. Auch bei der Betrachtung der Treppengeschwindigkeit abwärts wurde ähnliches Ergebnis von $654.274$ berechnet. Die Betrachtung zweier Parameter (Ebenengeschwindigkeit und Körpergröße) ergab sogar den Wert $3.27621*10^7$.
Diese schlechte Konditionierung erfolgt durch Rechenfehlern, die bei numerischen Berechnungen erfolgen. In diesen Berechnungen wurden 6 bis 7 ($6.1296$, $6.48353$) bzw. $17.3048$ Dezimalstellen verloren. Diese Zahlen wurden mit wurden mit dem natürlichen Logarithmus (zur Basis e) der Kondition bestimmt.

Wenn man die Konditionierung mit der QR-Zerlegung betrachtet, welche auch in Wirklichkeit von Mathematica verwendet wird, wurde eine wesentlich bessere Konditionierung ermittelt.

Die Betrachtung der Konditionierung der oberen Dreiecksmatrix R ergab bei der Treppengeschwindigkeit aufwärts nun einen Wert von $29.6338$. Hierbei wurden auch weniger verlorene Dezimalstellen festgestellt. Diese betrugen nur noch $3.38891$. Bei der Treppengeschwindigkeit abwärts wurde eine Konditionierung von $27.9171$ und $7164.48$ bei zwei Parametern (Verlorene Dezimalstellen bei $3.32924$ bzw. $8.87689$). Bei einem Parameter - der Ebenengeschwindigkeit - ist die Konditionierung nun deutlich besser. Für zwei Parameter (Körpergröße und Ebenengeschwindigkeit) lässt sich schließen, dass für diese Auswertung das Problem tatsächlich schlecht Konditioniert ist und die Aussagekraft der Ergebnisse bei dieser Untersuchung in Frage gestellt werden kann.

Dadurch, dass Mathematica die QR-Zerlegung benutzt sind diese Ergebnisse auch wesentlich aussagekräftiger als die erste Untersuchung der Konditionierung.



\section{Ergebnisse}
\section{Ermitteltes Modell}

\section{Vergleich mit Daten aus 2012}
\subsection{Überprüfung auf Normalverteilung}
\subsection{Lineare Regression}
\subsection{Vergleich}

\section{Verbund von alten und neuen Daten}

\section{Fazit}

\end{document}