\input{header}

\title{Zellulärer Zustandsautomat}
\providecommand{\subtitle}[1]{}
\subtitle{3. Projekt zu Modellierung und Simulation}
\author{Daniel Graf, Dimitrie Diez, Arne Schöntag, Peter Müller}
\date{}

\begin{document}

\maketitle

\tableofcontents
\section{Einführung}
Im Zuge der ersten Studienarbeit wurde das Laufverhalten von Probanden in der Ebene und auf der Treppe untersucht. Basierend auf den Erkenntnissen bezüglich der individuellen Wunschgeschwindigkeiten werden in dieser Studienarbeit Personenbewegungen in der Ebene simuliert. Hierfür wird ein zellulärer Zustandsautomat implementiert. Mit Hilfe solcher Simulationen können, beispielsweise im Zuge von Gebäudeplanungen unterschiedliche Gebäude- und Raumgestaltungen simuliert und hinsichtlich schnellster Räumungen im Krisenfall optimiert werden. 

\section{Beschreibung des Modells}

\section{Anforderungen/Requirements}
\label{requirements}

\section{Softwaredesign}

Der Aufbau der Anwendung wurde im Team diskutiert und anschließend mittels UML spezifiziert.


\section{Softwaretest}

\subsection{Verifikation}

\subsection{Validation}

\section{Ausblick und Fazit}


%	REFERENCE LIST
%----------------------------------------------------------------------------------------
%\input{References}
\bibliography{Bibliographie}
\bibliographystyle{plain}

\end{document}
