\documentclass[]{scrartcl}
\usepackage{lmodern}
\usepackage{amssymb,amsmath}
\usepackage{ifxetex,ifluatex}
\usepackage{fixltx2e} % provides \textsubscript
\ifnum 0\ifxetex 1\fi\ifluatex 1\fi=0 % if pdftex
  \usepackage[T1]{fontenc}
  \usepackage[utf8]{inputenc}
\else % if luatex or xelatex
  \ifxetex
    \usepackage{mathspec}
  \else
    \usepackage{fontspec}
  \fi
  \defaultfontfeatures{Ligatures=TeX,Scale=MatchLowercase}
\fi
% use upquote if available, for straight quotes in verbatim environments
\IfFileExists{upquote.sty}{\usepackage{upquote}}{}
% use microtype if available
\IfFileExists{microtype.sty}{%
\usepackage{microtype}
\UseMicrotypeSet[protrusion]{basicmath} % disable protrusion for tt fonts
}{}
\usepackage{hyperref}
\hypersetup{unicode=true,
            pdftitle={Angabe},
            pdfauthor={Team\ldots{}},
            pdfborder={0 0 0},
            breaklinks=true}
\urlstyle{same}  % don't use monospace font for urls
\IfFileExists{parskip.sty}{%
\usepackage{parskip}
}{% else
\setlength{\parindent}{0pt}
\setlength{\parskip}{6pt plus 2pt minus 1pt}
}
\setlength{\emergencystretch}{3em}  % prevent overfull lines
\providecommand{\tightlist}{%
  \setlength{\itemsep}{0pt}\setlength{\parskip}{0pt}}
\setcounter{secnumdepth}{5}
% Redefines (sub)paragraphs to behave more like sections
\ifx\paragraph\undefined\else
\let\oldparagraph\paragraph
\renewcommand{\paragraph}[1]{\oldparagraph{#1}\mbox{}}
\fi
\ifx\subparagraph\undefined\else
\let\oldsubparagraph\subparagraph
\renewcommand{\subparagraph}[1]{\oldsubparagraph{#1}\mbox{}}
\fi

\usepackage{graphicx}
\usepackage{array}
\usepackage{ragged2e}
\usepackage[section]{placeins}
\makeatletter
\AtBeginDocument{%
  \expandafter\renewcommand\expandafter\subsection\expandafter{%
    \expandafter\@fb@secFB\subsection
  }%
}
\makeatother

\title{Zellulärer Zustandsautomat}
\providecommand{\subtitle}[1]{}
\subtitle{3. Projekt zu Modellierung und Simulation}
\author{Daniel Graf, Dimitrie Diez, Arne Schöntag, Peter Müller}
\date{}

\begin{document}

\maketitle

\tableofcontents
\section{Einführung}
Im Zuge der ersten Studienarbeit wurde das Laufverhalten von Probanden in der Ebene und auf der Treppe untersucht. Basierend auf den Erkenntnissen bezüglich der individuellen Wunschgeschwindigkeiten werden in dieser Studienarbeit Personenbewegungen in der Ebene simuliert. Mit Hilfe solcher Simulationen können, beispielsweise im Zuge von Gebäudeplanungen unterschiedliche Gebäude- und Raumgestaltungen simuliert und hinsichtlich schnellster Räumungen im Krisenfall optimiert werden. 

\section{Beschreibung des Modells}
Für die Simulation der Personenbewegungen wird ein zellulärer Zustandsautomat implementiert. Jede Zelle kann entweder leer oder durch eine Person oder ein Hindernis besetzt sein. Sie kann außerdem ein Ziel oder eine Quelle enthalten. Ein Hindernis kann von keiner Person betreten werden. Die Personen versuchen im Laufe der Simulation von der Quelle (Startposition) zum Ziel zu gelangen. Hierbei können sie sich in der sog. Moore-Umgebung bewegen. Es kann jede Zelle betreten werden, die frei ist und eine Nachbarzelle der aktuellen Zelle ist. Als Nachbarzelle wird jede Zelle bezeichnet, welche mit der aktuellen Zelle eine Kante oder Ecke teilt. \\
Die Ziele werden als attraktiv für die Personen angesehen. Die Personen steigern ihren persönlichen Nutzen, je näher sie dem Ziel kommen. In der Realität fühlt sich eine Person jedoch unwohl, wenn ihr eine fremde Person zu nahe kommt. Im Modell wird dies dadurch berücksichtigt, dass sich der Nutzen einer Person verringert, wenn sie einer anderen zu nahe kommt. Um die Simulation möglichst realistisch zu gestalten, bewegen sich die Personen mit einer individuellen Wunschgeschwindigkeit. Diese wird aus den Ergebnissen der ersten Studienarbeit ermittelt. \\

Das erläuterte Modell gibt die einzelnen Personenbewegungen aus und visualisiert diese zusätzlich. Die Bewegung der Personen wird in drei getrennten Simulationen mit jeweils unterschiedlichen Algorithmen berechnet. Zunächst wird für jedes Feld der negative euklidische Abstand als Zielnutzen berechnet. Jede Person versucht durch die Wahl des umliegenden Feldes mit dem maximalen Wert seinen Nutzen zu steigern. \\
In einem zweiten Modell erfolgt die Ermittlung der Personenbewegung mit Hilfe des Floor-Flooding, basierend auf dem Dijkstra Algorithmus. Jede Person wählt von den anliegenden Feldern das Feld, welches die geringste Anzahl an notwendigen Bewegungen bis zum Ziel hat. \\
In einem dritten Modell erfolgt die Ermittlung der Personenbewegung ebenfalls mit Hilfe des Floor-Flooding. Als Grundlage dient hierbei jedoch die Lösung der Eikonal-Gleichung, welche die Ausbreitung einer Welle beschreibt. Hierfür wird der Fast-Marching Algorithmus verwendet.  \\
Im Laufe der Studienarbeit werden die beschriebenen Modelle anhand unterschiedlicher Tests verglichen und ausgewertet. Eine detaillierte Erläuterung der Tests ist im folgenden Kapitel \ref{requirements} beschrieben.


\section{Anforderungen/Requirements}
\label{requirements}

\section{Softwaredesign}

Der Aufbau der Anwendung wurde im Team diskutiert und anschließend mittels UML spezifiziert.


\section{Softwaretest}

\subsection{Verifikation}

\subsection{Validation}

\section{Ausblick und Fazit}


%	REFERENCE LIST
%----------------------------------------------------------------------------------------
%\input{References}
\bibliography{Bibliographie}
\bibliographystyle{plain}

\end{document}
