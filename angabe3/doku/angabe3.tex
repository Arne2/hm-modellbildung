\documentclass[]{scrartcl}
\usepackage{lmodern}
\usepackage{amssymb,amsmath}
\usepackage{ifxetex,ifluatex}
\usepackage{fixltx2e} % provides \textsubscript
\ifnum 0\ifxetex 1\fi\ifluatex 1\fi=0 % if pdftex
  \usepackage[T1]{fontenc}
  \usepackage[utf8]{inputenc}
\else % if luatex or xelatex
  \ifxetex
    \usepackage{mathspec}
  \else
    \usepackage{fontspec}
  \fi
  \defaultfontfeatures{Ligatures=TeX,Scale=MatchLowercase}
\fi
% use upquote if available, for straight quotes in verbatim environments
\IfFileExists{upquote.sty}{\usepackage{upquote}}{}
% use microtype if available
\IfFileExists{microtype.sty}{%
\usepackage{microtype}
\UseMicrotypeSet[protrusion]{basicmath} % disable protrusion for tt fonts
}{}
\usepackage{hyperref}
\hypersetup{unicode=true,
            pdftitle={Angabe},
            pdfauthor={Team\ldots{}},
            pdfborder={0 0 0},
            breaklinks=true}
\urlstyle{same}  % don't use monospace font for urls
\IfFileExists{parskip.sty}{%
\usepackage{parskip}
}{% else
\setlength{\parindent}{0pt}
\setlength{\parskip}{6pt plus 2pt minus 1pt}
}
\setlength{\emergencystretch}{3em}  % prevent overfull lines
\providecommand{\tightlist}{%
  \setlength{\itemsep}{0pt}\setlength{\parskip}{0pt}}
\setcounter{secnumdepth}{5}
% Redefines (sub)paragraphs to behave more like sections
\ifx\paragraph\undefined\else
\let\oldparagraph\paragraph
\renewcommand{\paragraph}[1]{\oldparagraph{#1}\mbox{}}
\fi
\ifx\subparagraph\undefined\else
\let\oldsubparagraph\subparagraph
\renewcommand{\subparagraph}[1]{\oldsubparagraph{#1}\mbox{}}
\fi

\usepackage{graphicx}
\usepackage{array}
\usepackage{ragged2e}
\usepackage[section]{placeins}
\makeatletter
\AtBeginDocument{%
  \expandafter\renewcommand\expandafter\subsection\expandafter{%
    \expandafter\@fb@secFB\subsection
  }%
}
\makeatother

\title{Zellulärer Zustandsautomat}
\providecommand{\subtitle}[1]{}
\subtitle{3. Projekt zu Modellierung und Simulation}
\author{Daniel Graf, Dimitrie Diez, Arne Schöntag, Peter Müller}
\date{}

\begin{document}

\maketitle

\tableofcontents
\section{Einführung}
Im Zuge der ersten Studienarbeit wurde das Laufverhalten von Probanden in der Ebene und auf der Treppe untersucht. Basierend auf den Erkenntnissen bezüglich der individuellen Wunschgeschwindigkeiten werden in dieser Studienarbeit Personenbewegungen in der Ebene simuliert. Mit Hilfe solcher Simulationen können, beispielsweise im Zuge von Gebäudeplanungen unterschiedliche Gebäude- und Raumgestaltungen simuliert und hinsichtlich schnellster Räumungen im Krisenfall optimiert werden. 

\section{Beschreibung des Modells}
Für die Simulation der Personenbewegungen wird ein zellulärer Zustandsautomat implementiert. Jede Zelle kann entweder leer oder durch eine Person oder ein Hindernis besetzt sein. Sie kann außerdem ein Ziel oder eine Quelle enthalten. Ein Hindernis kann von keiner Person betreten werden. Die Personen versuchen im Laufe der Simulation von der Quelle (Startposition) zum Ziel zu gelangen. Hierbei können sie sich in der sog. Moore-Umgebung bewegen. Es kann jede Zelle betreten werden, die frei ist und eine Nachbarzelle der aktuellen Zelle ist. Als Nachbarzelle wird jede Zelle bezeichnet, welche mit der aktuellen Zelle eine Kante oder Ecke teilt. \\
Die Ziele werden als attraktiv für die Personen angesehen. Die Personen steigern ihren persönlichen Nutzen, je näher sie dem Ziel kommen. In der Realität fühlt sich eine Person jedoch unwohl, wenn ihr eine fremde Person zu nahe kommt. Im Modell wird dies dadurch berücksichtigt, dass sich der Nutzen einer Person verringert, wenn sie einer anderen zu nahe kommt. Um die Simulation möglichst realistisch zu gestalten, bewegen sich die Personen mit einer individuellen Wunschgeschwindigkeit. Diese wird aus den Ergebnissen der ersten Studienarbeit ermittelt. \\

Das erläuterte Modell gibt die einzelnen Personenbewegungen aus und visualisiert diese zusätzlich. Die Bewegung der Personen wird in drei getrennten Simulationen mit jeweils unterschiedlichen Algorithmen berechnet. Zunächst wird für jedes Feld der negative euklidische Abstand als Zielnutzen berechnet. Jede Person versucht durch die Wahl des umliegenden Feldes mit dem maximalen Wert seinen Nutzen zu steigern. \\
In einem zweiten Modell erfolgt die Ermittlung der Personenbewegung mit Hilfe des Floor-Flooding, basierend auf dem Dijkstra Algorithmus. Jede Person wählt von den anliegenden Feldern das Feld, welches die geringste Anzahl an notwendigen Bewegungen bis zum Ziel hat. \\
In einem dritten Modell erfolgt die Ermittlung der Personenbewegung ebenfalls mit Hilfe des Floor-Flooding. Als Grundlage dient hierbei jedoch die Lösung der Eikonal-Gleichung, welche die Ausbreitung einer Welle beschreibt. Hierfür wird der Fast-Marching Algorithmus verwendet.  \\
Im Laufe der Studienarbeit werden die beschriebenen Modelle anhand unterschiedlicher Tests verglichen und ausgewertet. Eine detaillierte Erläuterung der Tests ist im folgenden Kapitel \ref{AnforderungenTest} beschrieben.


\section{Anforderungen/Requirements}
\label{requirements}

\subsection{Allgemeines}
Es soll ein zellulärer Zustandsautomat implementiert werden. Die Software soll Personenbewegungen in der Ebene simulieren. Die unterschiedlichen Szenarien (Zustand der Zellen zu Beginn der Simulation), wie beispielsweise die Position der Hindernisse werden der Software übergeben. Diese sind im Unterkapitel \ref{AnforderungenTest} näher erläutert. Es wird davon ausgegangen, dass sich zu Beginn der Simulation alle Personen im Startbereich (es soll möglich sein mehrere Startbereiche anzulegen) befinden.

\subsection{Zellen}
Das gesamte Feld (zweidimensionaler Bereich indem sich die Personen bewegen können) soll in Zellen eingeteilt werden. Die Größe des Feldes und die Anzahl an Zellen sollen der Software als Parameter übergeben werden.\\
Jede Zelle soll quadratisch sein. Jede Zelle kann verschiedene Zustände haben. Sie kann entweder leer oder besetzt sein. Ist eine Zelle besetzt, wird noch weiter unterschieden, ob sie durch ein Hindernis, das Ziel- (Personensenke) oder Startfeld (Personenquelle) oder durch eine Person besetzt ist.

\subsection{Hindernisse}
Es soll davon ausgegangen werden, dass jedes Hindernis eine oder mehrere Zellen ganz belegt. Hindernisse können von den Personen nicht betreten werden. Falls sich ein Hindernis zwischen einer Person und ihrem Ziel befindet, ist die Person gezwungen um das Hindernis herum zu gehen. \\
In der Realität versucht eine Person einem Hindernis möglichst frühzeitig auszuweichen. Sie senkt somit ihren Nutzen, je näher sie einem Hindernis kommt. Dies wird im weiteren Verlauf dieser Studienarbeit nicht berücksichtigt.

\subsection{Ziele}
Durch das entsprechende Testszenario werden ein oder mehrere Ziele definiert. Ziele werden als attraktiv für die Personen angesehen. Jede Person erhöht ihren Nutzen, wenn sie sich dem Ziel nähert. Befinden sich mehrere Ziele im Feld, wird die Wahl des Ziels für jede Person abhängig vom Betriebsmodus bestimmt. 

\subsection{Personen}
Jede Person soll in ihrem Startfeld starten und sich im Laufe der Simulation von Zelle zu Zelle auf ihr Ziel hin bewegen und dadurch ihren persönlichen Nutzen, je näher sie dem Ziel kommt, steigern. Jede Person kann sich in der Moore-Umgebung (8 mögliche Bewegungsrichtungen) bewegen. Die möglichen Bewegungsrichtungen sind in Abbildung \ref{fig:MooreUmgebung} rot markiert.
\begin{figure}[htpb]
	\centering
	\includegraphics[width=0.3\textwidth]{abbildungen/MooreUmgebung.pdf}
	\caption{Moore-Umgebung}
	\label{fig:MooreUmgebung}
\end{figure}
\\
Die Personen können nur freie Zellen betreten. Jede Person wählt genau die Zelle als nächste Zelle, welche ihr den steilsten Nutzenanstieg ermöglicht. Der Nutzen der verschiedenen Zellen wird abhängig vom Betriebsmodus unterschiedlich berechnet. Darüber hinaus soll der Nutzen durch die Position der anderen Personen beeinflusst werden. \\

 %TODO --> Punkt 3 fehlt noch, Diskussion und genauere Beschreibung auch 	
Der Nutzen einer Person wird verringern, wenn sie anderen Personen zu nahe kommt. In der Simulation soll folgende Funktion verwendet werden. 

\begin{displaymath}
u_{p}(d) = 
\left\{ 
\begin{array}{cc}
-h \exp(\frac{1}{(\frac{d}{w})^2 -1}) & \;\;\;\;\;\;\;\;\;\;\; falls\, |d| < w  \\
0 & sonst  
\end{array} 
\right.
\end{displaymath}
Hierbei wird mit $d$ der Abstand zur nächsten Person bezeichnet. $w$ und $h$ stehen für die Stärke und Reichweite der Funktion. 
 
\subsection{Geschwindigkeit der Personen}
Jede Person soll eine individuelle Wunschgeschwindigkeit besitzen. Sie bewegt sich bei freier Bahn mit dieser Geschwindigkeit fort. Die Wunschgeschwindigkeiten sollen als normalverteilt angesehen werden. Als Mittelwert und Varianz sollen die errechneten Werte aus der ersten Studienarbeit verwendet werden.
\\%TODO --> Genaue Werte aus erster Studienarbeit einfügen

\subsection{Zeitmodellierung}
Die Applikation unterliegt keinen harten Echtzeitanforderungen. Die Simulation soll für die gesamte Simmulationsdauer so schnell wie möglich durchgeführt werden. Es soll keine Globale Uhr modelliert werden.

\subsection{Parameter}
Die Tabelle \ref{tab:parameter} zeigt alle möglichen Parameter, welche die Software steuern. Alle Parameter sind optional.
\begin{table}[htpb]
	\centering
	\begin{tabular}{lll}
		Parameter & Beschreibung  &  Default\\ \hline
		- & - & - \\
	\end{tabular}
	\caption{Parameter und Defaultwerte}
	\label{tab:parameter}
\end{table}
	%TODO --> Parameter der Software ermitteln und eintragen
	
\subsection{Berechnungen}
Die Aufgabe der Software ist die Simulation von Personenbewegungen. Für möglichst genaue Zeitangaben wird die Java Klasse BigDecimal verwendet. Für die Division werden 32 Nachkommastellen berücksichtigt und als Rundungsmodus wird Half\_Even verwendet, da dieser kumulative Fehler bei sich wiederholenden Berechnungen minimiert. Die anschließende Berechnung und Aufbereitung der Daten (insebesondere für den Test \ref{Anforderungen:RimeaTest}) erfolgt mit Mathematica.

\subsection{Reports}
%TODO --> Ermitteln was alles ausgegeben wird

\subsection{Betriebsmodi}
\label{AnforderungenBetriebsmodi}

Die Software hat 3 Betriebsmodi. Abhängig vom Betriebsmodus wird der Nutzen bzw. die Attraktivität der einzelnen Zellen, bezogen auf das/die Ziel(e) mit anderen Algorithmen berechnet. \\
\\
Betriebsmodus 1: Der Nutzen bzw. die Attraktivität jeder Zelle entspricht dem negativen euklidischen Abstand. In diesem Betriebsmodus können keine Hindernisse umgangen werden. Jede Person versucht den Nutzen zu maximieren und wählt jeweils die Zelle mit maximalen Wert aus den umliegenden aus.\\
\\
Betriebsmodus 2: Der Nutzen bzw. die Attraktivität jeder Zelle wird durch Floor-Flooding, basierend auf dem Dijkstra Algorithmus, berechnet. Für jede Zelle wird die geringste Anzahl an notwendigen Bewegungen bis zum Ziel ermittelt. \\
\\
Betriebsmodus 3: Analog zum Betriebsmodus 2 wird auch in diesem Betriebsmodus der Nutzen bzw. die Attraktivität jeder Zelle mittels Floor-Flooding berechnet. In diesem Betriebsmodus wird jedoch der Fast-Marching Alforithmus (Lösung der Eikonal-Gleichung) als Grundlage verwendet.

\subsection{Werte für die Auswertung}
Alle in der Tabelle \ref{tab:WerteAuswertung} aufgeführten Werte sollen auf Basis der Simulation ermittelt und ausgewertet werden. Einige Werte werden nur für bestimmte Tests benötigt.
\begin{table}[htpb]
	\centering
	\begin{tabular}{lll}
		Ermittelte Werte & Beschreibung & Wann benötigt\\ \hline
		mittlere Geschwindigkeit & durchschnittliche Geschwindigkeit der Personen & Rimea-Test 4 \\
		
		Fluss (Personen/Meter/s) & Anzahl der Personen mit Geschwindigkeit > 1m/s & Rimea-Test 4\\
		Simulationsdauer & Dauer bis der Test beendet ist & Hühner Test, \\&& Evakuierung (2 Türen), \\&& Evakuierung (4 Türen)\\
	\end{tabular}
	\caption{Werte für die Auswertung}
	\label{tab:WerteAuswertung}
\end{table}

\subsection{Test der Implementierung}
%TODO --> Beschreiben was alles getestet wurde

\subsection{Testszenarien für die Simulation}
\label{AnforderungenTest}
%TODO --> Bilder mit jeweiligen Testszenarien einfügen
%TODO --> Definieren mit welcher Dichte, wie viele Personen etc der Test durchgeführt werden muss
Die implementierten Algorithmen (vgl. Betriebsmodi im Unterkapitel \ref{AnforderungenBetriebsmodi} ) sollen anhand unterschiedlicher Testszenarien verglichen und verifiziert werden. 
 
\subsubsection{Freier Fluss}
Im Zuge dieses Tests soll überprüft werden, ob sich Personen bei freier Bahn (keine Hindernisse zwischen Start und Ziel) mit ihrer jeweiligen Wunschgeschwindigkeit auf das Ziel hin bewegen. Des weiteren soll überprüft werden, ob die Person den kürzesten Weg verwendet. Der Test soll mit allen 3 Betriebsmodi durchgeführt werden.

\subsubsection{Hühnertest}
Zwischen Start und Ziel soll ein U-förmiges Hindernis (Öffnung in Richtung des Startes) eingefügt werden. Abhängig vom Betriebsmodus wird ein unterschiedliches Ergebnis erwartet:\\
Betriebsmodus 1: Die Personen sollen sich auf dem kürzesten Weg auf das Ziel zubewegen. Sobald sie das Hindernis erreicht haben, sollen sie stehen bleiben. Ein erfolgreiches absolvieren des Tests wird nicht erwartet. \\
Betriebsmodi 2 und 3: Die Personen sollen das Hindernis umgehen und das Ziel erreichen. Bei einem \glqq Feststecken im Hindernis\grqq (wie im Betriebsmodus 1 gefordert) wird der Test als fehlgeschlagen gewertet.
 
\subsubsection{Evakuierung eines Raumes mit 2 Türen}
\label{Anforderungen:Evakuierung2}
Die Personen befinden sich zu Beginn in einen von Hindernissen umgebenen, quadratischen Raum. An 2 Seiten sind Öffnungen platziert (freie Zellen). Hinter diesen Öffnungen befindet sich jeweils ein Ziel. Im Laufe des Tests sollen die Personen durch die beiden Engstellen zum Ziel gelangen. \\
Dieser Test soll mit den Betriebsmodi 2 und 3 in unterschiedlichen Versionen durchgeführt werden. Zunächst sollen sich die Türen an benachbarten Seiten befinden. Hierbei sollen sie einmal mit minimalen und einmal mit maximalen Abstand zueinander plaziert werden. Anschließend sollen die Türen an gegenüberliegenden Seiten plaziert werden. Auch hierbei sollen sie einmal direkt gegenüber (mittig) und einmal mit maximalen Abstand zueinander (versetzt) angeordnet werden. Unterschiede hinsichtlich der Evakuierungsdauer sind festzuhalten.

\subsubsection{Evakuierung eines Raumes mit 4 Türen}
Dieser Testfall entsprich der in Kapitel \ref{Anforderungen:Evakuierung2} beschriebenen Evakuierung mit 2 Türen. In diesem Fall befindet sich jedoch an jeder Raumseite eine Tür. Die Türen sollen sich in einer ersten Version mittig bezogen auf die jeweilige Raumseite befinden. Darüber hinaus sollen in einer weiteren Version die Auswirkungen ermittelt werden, wenn jeweils die 2 Türen an benachbarten Seiten mit minimalen Abstand zueinander plaziert werden. Unterschiede hinsichtlich der Evakuierungsdauer sind festzuhalten.

\subsubsection{Fundamentaldiagramm - Rimea-Test 4}
\label{Anforderungen:RimeaTest}

\section{Softwaredesign}

Der Aufbau der Anwendung wurde im Team diskutiert und anschließend mittels UML spezifiziert.


\section{Softwaretest}

\subsection{Verifikation}

\subsection{Validation}

\section{Ausblick und Fazit}

%----------------------------------------------------------------------------------------
%\input{References}
\bibliography{Bibliographie}
\bibliographystyle{plain}

\end{document}
